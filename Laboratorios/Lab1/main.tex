\documentclass[12pt]{article}

\usepackage{sbc-template}
\usepackage{graphicx,url}
\usepackage[utf8]{inputenc}
\usepackage[brazil]{babel}
\usepackage[latin1]{inputenc}  

     
\sloppy

\title{TEMPLATE}

\author{Gabriel S. -- DRE: XXXXXXXXX\\ \and Leo -- DRE: XXXXXXXXX\\ \and Lucas I. -- DRE: XXXXXXXXX}

\address{PESC -- Universidade Federal do Rio de Janeiro (UFRJ)
  \email{leonardongc@poli.ufrj.br}
}

\begin{document} 

\maketitle

\section{}
browser - http 1.1; servidor idem
\section{}
inglês - global; inglês - EUA
\section{}
IP-source: 10.0.0.101; IP-destination: 146.164.70.1
\section{}
 status: 304
\section{}
date 26/04/2023 15:40 GMT
\section{}
415 bytes de quadro
\section{}
Não notamos diferença
\section{}
No primeiro GET (com o cache e histórico limpo), não se observou tal sentença.
\section{}
Em resposta ao GET inicial o servidor retornou o conteúdo do arquivo. Isso pode ser constatado mediante a análise da aba Hypertext Transfer Protocol e Line-Based Text Data. No primeiro pode-se observar que o conteúdo é chamado e mostrado no segundo. 
\section{}
Após a atualização da página -- gerando um segundo GET -- foi possível visualizá-la: If-Modified-Since: Tue, 16 May 2023 05:59:01 GMT\\r\\n.
\section{}
HTTP/1.1 304 Not Modified. Não, não foram retornados o conteúdo dos arquivos, visto que  os dados foram carregados do cache browser. Isso se confirma, seguindo a linha de raciocínio da questão 9, no GET posterior, o Line-Based Text Data não apareceu, indicando que a página foi carregada do cache.
\section{}
dois http get, pckt \#802
\section{}
pckt \#802, 
\section{}
status 200 OK (text/html)
\section{}
5 segmentos
\section{}
3 GETs: 
\section{}
Sequencialmente olhando os timestamps do TCP
\section{}
401 UNAUTHORIZED
\section{}
Campo de autorização


\end{document}
